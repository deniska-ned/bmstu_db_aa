\chapter*{Введение}
\addcontentsline{toc}{chapter}{Введение}

    В данной лабораторной работе будет рассмотрен последовательный и параллельный алгоритм лексикографического сравнения строк.

    Многопоточность – способность центрального процессора (ЦПУ) или одного ядра в многоядерном процессоре одновременно выполнять несколько процессов или потоков, соответствующим образом поддерживаемых операционной системой. 

    Этот подход отличается от многопроцессорности, так как многопоточность процессов и потоков совместно использует ресурсы одного или нескольких ядер: вычислительных блоков, кэш-памяти ЦПУ или буфера перевода с преобразованием.

    В тех случаях, когда многопроцессорные системы включают в себя несколько полных блоков обработки, многопоточность направлена на максимизацию использования ресурсов одного ядра, используя параллелизм на уровне потоков, а также на уровне инструкций.

    Поскольку эти два метода являются взаимодополняющими, их иногда объединяют в системах с несколькими многопоточными ЦП и в ЦП с несколькими многопоточными ядрами.

    Многопоточная парадигма стала более популярной с конца 1990-х годов, поскольку усилия по дальнейшему использованию параллелизма на уровне инструкций застопорились.

    Целью работы является исследование паралельного алгоритма лексикографического сравнения строк.

    Для достижения поставленной цели необходимо выполнить следующие задачи:

    \begin{enumerate}
        \item Изучение алгоритмов лексикографического сравнения строк
        \item Реализация последовательного алгоритма лексикографического сравнения строк
        \item Реализация параллельного алгоритма лексикографического сравнения строк
        \item Экспериментальное сравнение временных характеристик реализованных алгоритмов
    \end{enumerate}
