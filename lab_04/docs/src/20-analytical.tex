\chapter{Аналитическая часть}

    В данном разделе описаны математические модели исследуемой области.

    \section{Описание задачи}

        Лексикографический порядок — отношение линейного порядка на множестве слов над некоторым упорядоченным алфавитом \( \sum \) . Своё название лексикографический порядок получил по аналогии с сортировкой по алфавиту в словаре. 
        
        \subsection*{Определение}
        
            Слово \( \alpha \) предшествует слову \( \beta \) (\( \alpha \) < \( \beta \)), если
            
            \begin{itemize}
                \item  либо первые \( m \) символов этих слов совпадают, а \( m+1 \)-й символ слова \( \alpha \) меньше (относительно заданного в \( \sum \) порядка) \( m + 1 \)-го символа слова \( \beta \) (например, АБАК < АБРАКАДАБРА, так как первые две буквы у этих слов совпадают, а третья буква у первого слова меньше, чем у второго);
                \item  либо слово \( \alpha \) является началом слова \( \beta \) (например, МАТЕМАТИК < МАТЕМАТИКА).
            \end{itemize}

        \subsection*{Примеры}

            Порядок слов в словаре. Предполагается, что буквы можно сравнивать, сравнивая их номера в алфавите. Например, следующие слова идут в лексикографическом порядке: А < АА < ААА < ААБ < ААВ < АБ < Б < … < ЯЯЯ.
    
            Естественный порядок на неотрицательных целых \(n\)-значных числах в любой позиционной системе счисления, записанных в разрядной сетке фиксированной длины (000, 001, 002, 003, 004, 005, …, 998, 999).
    
    \section*{Вывод}
    \addcontentsline{toc}{section}{Вывод}
    
        В результате были рассмотрены и описаны математические модели следующих алгоритмов:
        
        \begin{itemize}
            \item Последовательный алгоритм лексикографического сравнения строк
            \item Параллельный алгоритм лексикографического сравнения строк
        \end{itemize}
        
        В качестве входных данных алгоритмы принимают 2 строки. Выходными данными алгоритма является число. Если число равно нулю, то строки равны, если число больше нуля, то первая строка больше второй иначе вторая строка больше первой.
        
        Ограничением для реализуемой программы является обработка строк только в кодах ascii.c