\chapter{Технологическая часть}

    В данном разделе приведены требования к программному обеспечению, средства реализации и листинги кода.
    
    \section{Выбор средств реализации}
        
        Для реализации библиотеки с реализациями алгоритмов мною был выбрал язык C++\cite{cppstd} стандарта 2014 года. Он был выбран в силу предоставления библиотеки работы с физическими потоками.
        
        Для юнит тестирования был выбран язык C++14, для использовая фрейворка GoogleTests\cite{gtests}.
    
    \section{Требования к программному обеспечению}
    
        К программе предъявляется ряд требований:
        
        \begin{itemize}
            \item На вход подаётся две регистрозависимые строки.
            \item На выходе — результат выполнения каждого из вышеуказанных алгоритмов.
        \end{itemize}
    
    \clearpage
    
    \section{Листинги кода}
    
        \lstinputlisting[
        language={C++},
        caption=Основной файл программы main.cpp,
        basicstyle=\scriptsize
        ]{./listings/main.cpp}
        
        \lstinputlisting[
        language={C++},
        caption=Последовательная реализация алгоритма,
        basicstyle=\scriptsize
        ]{./listings/nonparallel.cpp}
        
        \newpage
        
        \lstinputlisting[
        language={C++},
        caption=Параллельная реализация алгоритма,
        basicstyle=\scriptsize
        ]{./listings/parallel.cpp}
        
    \section*{Вывод}
    \addcontentsline{toc}{section}{Вывод}
    
        Были реализованы алгоритмы последовательного лексикографического сравнения строк и параллельного лексикографического сравнения строк.