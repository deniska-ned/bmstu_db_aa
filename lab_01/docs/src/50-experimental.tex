\chapter{Экспериментальная часть}

    В данном разделе будет проведено функциональное тестирование разработанного программного обеспечения. Так же будет произведено измерение временных характеристик и характеристик по памяти каждого из реализованных алгоритмов.

    \section{Технические характеристики}
    
        Технические характеристики устройства, на котором выполнялось исследование:
        
        \begin{itemize}
            \item Процессор: Intel(R) Core(TM) i5-8250U CPU @ 1.60GHz \cite{intel}
            \item Оперативная память: 8 Gb
            \item Операционная система: elementary OS 6 Odin \cite{elemos}
        \end{itemize}
    
    \section{Тестирование}
    
        Юнит тестирование проводилось при помощи фрэймворка GoogleTests. Были выполнены следующие тесты
        
        \begin{table}[]
            \centering
            \caption{Функциональные тесты}
            \begin{tabular}{|c|c|c|c|}
\hline \textbf{Строка 1} & \textbf{Строка 2} & \textbf{Левенштейн} & \textbf{Дамерау-Левенштейн}  \\
\hline CONNECT & CONEHEAD & 4 &  \\
\hline & CONEHEAD & 8 & 8 \\
\hline &  & 0 & 0 \\
\hline deniska & deniska & 0 & 0 \\
\hline abc & bc & 1 & 1 \\
\hline abc & ac & 1 & 1 \\
\hline abc & ab & 1 & 1 \\
\hline abc & xbc & 1 & 1 \\
\hline abc & axc & 1 & 1 \\
\hline abc & abx & 1 & 1 \\
\hline CONNECT & CNNNETC &  & 2 \\
\hline abcd & badc &  & 2 \\
\hline
            \end{tabular}
            \label{tab:my_label}
        \end{table}
        
        
        При проведении функционального тестирования, полученные результаты работы программы совпали с ожидаемыми. Таким образом, функциональное тестирование пройдено успешно.
    
    \section{Временные характеристики}
    
        Результаты замеров по результатам экспериментов приведены в Таблице \ref{tab:time}. В данной таблице для значений, для которых тестирование не выполнялось, в поле результата находится ” - ”.
    
        \begin{table}[]
            \centering
            \caption{Замер времени для строк разной длины}
            \resizebox{\columnwidth}{!}{% 
            \begin{tabular}{|c|c|c|c|c|}
\hline \multirow{3}{*}{\textbf{Длина строк}} & \multicolumn{4}{c|}{\textbf{Время, сек}} \\
\cline{2-5} & \multicolumn{2}{c|}{\textbf{Левенштейн}} & \multicolumn{2}{c|}{\textbf{Дамерау-Левенштейн}} \\
\cline{2-5} & \textbf{Рекурсивный} & \textbf{Итеративный} & \textbf{Рекурсивный} & \textbf{Итеративный} \\
\hline    1 &   3e-08 & 5.3e-08 & 3.6e-08 & 6.6e-08 \\
\hline    2 & 6.7e-05 & 3.1e-07 & 6.6e-05 & 3.8e-07 \\
\hline    4 & 8.1e-05 & 3.2e-07 & 6.6e-05 & 3.8e-07 \\
\hline    6 & 6.6e-04 & 4.2e-07 & 6.5e-05 & 3.8e-07 \\
\hline    8 &  0.0019 & 5.2e-07 &   0.002 & 6.7e-07 \\
\hline   16 &       - &   2e-06 &       - & 2.7e-06 \\
\hline   32 &       - & 8.8e-06 &       - & 1.3e-05 \\
\hline   64 &       - & 3.5e-05 &       - & 4.4e-05 \\
\hline  128 &       - & 0.00012 &       - & 0.00016 \\
\hline  256 &       - & 0.00048 &       - & 0.00064 \\
\hline  512 &       - &  0.0019 &       - &  0.0025 \\
\hline 1024 &       - &  0.0075 &       - &    0.01 \\
\hline
            \end{tabular}%
            }
            \label{tab:time}
        \end{table}
    
   Отдельно сравним итеративные алгоритмы поиска расстояний Левенштейна и Дамерау– Левенштейна. Сравнение будет производится на основе данных, представленных в Таблице \ref{tab:time}. Результат можно увидеть на Рисунке \ref{plt:time_cmp_l_dl}. 
        
    \begin{figure}[!h]
      \centering
      \begin{tikzpicture}
        \begin{axis}[
          axis lines=left,
          xlabel=Длина строк,
          ylabel={Время, мс},
          legend pos=north west,
          ymajorgrids=true
        ]
          \addplot table[x=strlen,y=leven_iter,col sep=comma] {datasheets/itr.csv};
          \addplot table[x=strlen,y=dleven_iter,col sep=comma] {datasheets/itr.csv};
          \legend{Левенштейн, Дамерау-Левенштейн}
        \end{axis}
      \end{tikzpicture}
      \captionsetup{justification=centering}
      \caption{Сравнение времени работы итеративных алгоритмов Левенштейна и Дамерау-Левенштейна}
      \label{plt:time_cmp_l_dl}
    \end{figure}
    
    При длинах строк менее 64 символов разница по времени между итеративными реализациями незначительна, однако при увеличении длины строки алгоритм поиска расстояния Левенштейна оказывается быстрее вплоть до полутора раз. Это обосновывается тем, что у алгоритма поиска расстояния Дамерау-Левенштейна задействуется дополнительная операция, которая замедляет алгоритм.
    
    Так же сравним рекурсивную и итеративную реализации алгоритма поиска расстояния Левенштейна. Данные представлены в Таблице \ref{tab:time} и отображены на Рисунке \ref{plt:time_cmp_rec_iter}.
   
    \begin{figure}[!h!]
      \centering
      \begin{tikzpicture}
        \begin{axis}[
          axis lines=left,
          xlabel=Длина строк,
          ylabel={Время, мс},
          legend pos=north west,
          ymajorgrids=true
        ]
          \addplot table[x=strlen,y=leven_rec,col sep=comma] {datasheets/all.csv};
          \addplot table[x=strlen,y=leven_iter,col sep=comma] {datasheets/all.csv};
          \legend{Рекурсивный, Итеративный}
        \end{axis}
      \end{tikzpicture}
      \captionsetup{justification=centering}
      \caption{Сравнение времени работы итеративного и рекурсивного алгоритмов Леверштейна}
      \label{plt:time_cmp_rec_iter}
    \end{figure}
    
    \clearpage
    
        
    \section{Сравнительный анализ алгоритмов}
    
        Приведенные характеристики показывают нам, что рекурсивная реализация алгоритма проигрывает по времени для всех тестовых данных. Например для длины строк длины 1 алгоритм работает в 1.7 раза дольше, для длины строк 4 - в 203 раза дольше, для длины строк 8 - 3500 раз дольше. Поэтому выбор рекурсивного алгоритма перед итеративным не оправдывает себя.
        
        Так как во время печати очень часто возникают ошибки связанные с транспозицией букв, алгоритм поиска расстояния Дамерау – Левенштейна является наиболее предпочтительным, не смотря на то, что он проигрывает по времени и памяти алгоритму Левенштейна.
        
        По аналогии с первым абзацем можно сделать вывод о том, что рекуррентный алгоритм поиска расстояния Дамерау - Левенштейна будет более затратным по времени по сравнению с итеративной реализацией алгоритма поиска расстояния Дамерау – Левенштейна с кешированием.
    
    \section*{Вывод}
    \addcontentsline{toc}{section}{Вывод}
    
        В данном разделе было произведено сравнение количества затраченного времени и пaмяти вышеизложенных алгоритмов. Наиболее затратным по времени оказался рекурсивный алгоритм нахождения расстояния Дамерау – Левенштейна.
        
        Рекурсивных алгоритм проигрывает по времени перед итеративным тем сильнее, чем больше длина строк. Для строк длины 1 разница незначительна: 1.7. Однако для строк большей длины, разность во времени отличается на порядки: для длины строк 4 - рекурсивный работает в 203 раза больше, а для длины строк 8 - в 3500 раз дольше. Однако, стоит учитывать дополнительные затраты по памяти, возникающие при использовании итеративных алгоритмов.