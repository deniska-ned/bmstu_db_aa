\chapter*{Заключение}
\addcontentsline{toc}{chapter}{Заключение}

    Алгоритмы поиска расстояний Левенштейна и Дамерау-Левенштейна являются самыми популярными алгоритмами, которые помогают найти редакторское расстояние.
    
    В результате выполнения данной лабораторной работы были изучены алгоритмы поиска расстояний Левенштейна \eqref{eq:D} и Дамерау – Левенштейна \eqref{eq:d}, построены схемы (\ref{fig:leven_iter}, \ref{fig:dleven_iter}), соответствующие данным алгоритмам, также разобраны рекурсивные алгоритмы (\ref{fig:leven_rec}, \ref{fig:dleven_rec}). Было проведено сравнение алгоритмов по памяти и по времени. Была написана программа, реализующая данных алгоритмы.
    
    Рекурсивная реализация алгоритма существенного проигрывает перед итеративным вариантом по времени. И эта разница тем сильнее, чем больше длина строк. Для строк длины 1 рекурсивный алгоритм работает в 1.7 раз дольше, для строк длины 4 - в 203 раза дальше, для строк длины 8 - в 3500 раз дольше.
    
    Алгоритм Левенштейна работает быстрее алгоритма в Дамерау - Левенштейна в среднем в 1.3 раза. Однако ошибка перестановки букв в строке так часта, что компенсирует данный проигрыш по времени, поэтому стоит предпочитать итеративный алгоритм поиска расстояния Дамерау - Левенштейна перед другими.
    