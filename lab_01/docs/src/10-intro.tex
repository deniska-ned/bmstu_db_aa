\chapter*{Введение}
\addcontentsline{toc}{chapter}{Введение}

Расстояние Левенштейна (редакционное расстояние, дистанция редактирования) — метрика, измеряющая по модулю разность между двумя последовательностями символов. Она определяется как минимальное количество односимвольных операций (а именно вставки, удаления, замены), необходимых для превращения одной последовательности символов в другую. В общем случае, операциям, используемым в этом преобразовании, можно назначить разные цены. Широко используется в теории информации и компьютерной лингвистике.

Впервые задачу поставил в 1965 году советский математик Владимир Левенштейн при изучении последовательностей \(0 − 1\) \cite{leven1965}, впоследствии более общую задачу для произвольного алфавита связали с его именем.

Расстояние Левентшейна используется:

\begin{itemize}
    \item в компьютерной лингвистике для проведения автозамены
    \item в биоинформатике для сравнения генов, хромосом, белков
    \item в утилите diff для определения изменений в текстовых файлов
\end{itemize}

Расстояние Дамерау — Левенштейна (названо в честь учёных Фредерика Дамерау и Владимира Левенштейна) — это мера разницы двух строк символов, определяемая как минимальное количество операций вставки, удаления, замены и транспозиции (перестановки двух соседних символов), необходимых для перевода одной строки в другую. Является модификацией расстояния Левенштейна: к операциям вставки, удаления и замены символов, определённых в расстоянии Левенштейна добавлена операция транспозиции (перестановки) символов.

Целью работы является исследование алгоритмов Левенштейна и Дамераy - Левенштейна.

Для достижения поставленной цели необходимо выполнить следующие задачи:

\begin{enumerate}
    \item изучение алгоритма Левенштейна и Демаран - Левенштейна
    \item реализация рекурсивных (без кеша) и итеративных (с кешем) алгоритмов Левенштейна и Дамераy - Левенштейна
    \item замер и сравнение процессорного времени затрачиваемых реализованными алгоритмами
    \item замер и сравнение используемой памяти реализованными алгоритмами
\end{enumerate}