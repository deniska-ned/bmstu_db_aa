\chapter{Технологическая часть}

    В данном разделе приведены требования к программному обеспечению, средства реализации и листинги кода.
    
    \section{Выбор средств реализации}
        
        Для реализации библиотеки с реализациями алгоритмов мною был выбрал язык C\cite{cstd} стандарта 1999 года. Он был выбран в силу отсутствия необходимости в использовании высокоуровневых абстракций.
        
        Для реализации клиента был выбран язык C расширения GNU 99 года, так как он предоставляет фукнкции чтения динамических строк.
        
        Для юнит тестирования был выбран язык C++14, для использовая фрейворка GoogleTests\cite{gtests}.
    
    \section{Требования к программному обеспечению}
    
        К программе предъявляется ряд требований:
        
        \begin{itemize}
            \item На вход подаётся две регистрозависимые строки.
            \item На выходе — результат выполнения каждого из вышеуказанных алгоритмов.
        \end{itemize}
    
    \clearpage
    
    \section{Листинги кода}
    
        \lstinputlisting[
        language={C},
        caption=Основной файл программы main.c,
        basicstyle=\scriptsize
        ]{./listings/main.c}
        
        \clearpage
        
        \lstinputlisting[
        language={C},
        caption=Рекурсивная функция нахождения расстояния Левенштейна,
        basicstyle=\scriptsize
        ]{./listings/leven_rec.c}
        
        \lstinputlisting[
        language={C},
        caption=Рекурсивная функция нахождения расстояния Дамерау - Левенштейна,
        basicstyle=\scriptsize
        ]{./listings/dleven_rec.c}
        
        \lstinputlisting[
        language={C},
        caption=Итеративная функция нахождения расстояния Левенштейна,
        basicstyle=\scriptsize
        ]{./listings/leven_iter.c}
        
        \clearpage
        
        \lstinputlisting[
        language={C},
        caption=Итеративная функция нахождения расстояния Дамерау - Левенштейна,
        basicstyle=\scriptsize
        ]{./listings/dleven_iter.c}
        
        \clearpage
        
        \lstinputlisting[
        language={C},
        caption=Вспомогательные функции,
        basicstyle=\scriptsize
        ]{./listings/support.c}
        
        \clearpage
        
        \lstinputlisting[
        language={C++},
        caption=Юнит тесты для алгоритма поиска расстояния Левенштейна. Часть 1,
        basicstyle=\scriptsize
        ]{./listings/test_leven_iter_01.cpp}
        
        \lstinputlisting[
        language={C++},
        caption=Юнит тесты для алгоритма поиска расстояния Левенштейна. Часть 2,
        basicstyle=\scriptsize
        ]{./listings/test_leven_iter_02.cpp}
        
    \section*{Вывод}
    \addcontentsline{toc}{section}{Вывод}
    
        Были реализованы алгоритмы: вычисления расстояния Левенштейна рекурсивно, с зaполнением кэша, а также вычисления расстояния Дамерау – Левенштейна рекурсивно и вычисления расстояния Дамерау – Левенштейна с заполнением кэша.