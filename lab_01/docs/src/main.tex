%% Преамбула TeX-файла

% 1. Стиль и язык
\documentclass[utf8x]{G7-32} % Стиль (по умолчанию будет 14pt)
\usepackage[T2A]{fontenc}
\usepackage[russian]{babel}
% Остальные стандартные настройки убраны в preamble.inc.tex.
\include{preamble.inc}

% Настройки листингов.
\include{listings.inc}

% Полезные макросы листингов.
\include{macros.inc}

%For titul
%--------------------------------------
\usepackage{graphicx}
\graphicspath{ {./images/} }
\usepackage{tabularx} % in the preamble
\usepackage[normalem]{ulem}
%--------------------------------------

%graphics
\usepackage{tikz,pgfplots}

\begin{document}

\begin{titlepage}
    \thispagestyle{empty}

    \noindent\begin{minipage}{0.05\textwidth}
        \includegraphics[scale=0.3]{bmstu}
    \end{minipage}
    \hfill
    \begin{minipage}{0.85\textwidth}\raggedleft
        \begin{center}
            \fontsize{10pt}{0.3\baselineskip}\selectfont \textbf{Министерство науки и высшего образования Российской Федерации \\ Федеральное государственное бюджетное образовательное учреждение \\ высшего образования \\ <<Московский государственный технический университет \\ имени Н. Э. Баумана \\ (национальный исследовательский университет)>> \\ (МГТУ им. Н. Э. Баумана)}
        \end{center}
    \end{minipage}

    \begin{center}
        \fontsize{12pt}{0.1\baselineskip}\selectfont
        \noindent\makebox[\linewidth]{\rule{\textwidth}{4pt}} \makebox[\linewidth]{\rule{\textwidth}{1pt}}
    \end{center}

    \begin{flushleft}
        \fontsize{12pt}{0.8\baselineskip}\selectfont

        ФАКУЛЬТЕТ \uline{
            \hfill
            Информатика и системы управления
            \hfill}

        КАФЕДРА \uline{\mbox{\hspace{4mm}}
            \hfill
            Программное обеспечение ЭВМ и информационные технологии
            \hfill}
    \end{flushleft}

    \vfill
    
    \begin{center}
        \fontsize{19pt}{\baselineskip}\selectfont
        \textbf{Отчет по лабораторной работе № 4} \\
        \textbf{по курсу "Анализ алгоритмов"}
    \end{center}

    \vfill
    
    \begin{flushleft}
        Студент \uline{\hfill Недолужко Денис Вадимович \hfill}
        
        Группа \uline{\hfill ИУ7-53Б \hfill}
        
        Название предприятия \uline{\hfill МГТУ им. Н. Э. Баумана, каф. ИУ7  \hfill}
        
        Тема \uline{\hfill Параллельное лексикографическое сравнение строк \hfill}
    \end{flushleft}
    
    \vfill
    
    \begin{flushleft}
    \begin{tabularx}{\textwidth}{Xcc}
      Студент& \uline{\hfill} &  \hfill\uline{Недолужко Д. В.}\\
      &\textsuperscript{\scriptsize{(Подпись, дата)}}&\textsuperscript{\scriptsize{(И.О. Фамилия)}}\\
      Руководитель курсовой работы & \uline{\hfill} & \uline{Волкова Л. Л.}\\
      &\textsuperscript{\scriptsize{(Подпись, дата)}}&\textsuperscript{\scriptsize{(И.О. Фамилия)}}\\
    \end{tabularx}
    \end{flushleft}
    
    \vfill
    
    \begin{center}
        \normalsize Москва \\
        \the\year ~г.
    \end{center}
\end{titlepage}

\frontmatter % выключает нумерацию ВСЕГО; здесь начинаются ненумерованные главы: реферат, введение, глоссарий, сокращения и прочее.

% Команды \breakingbeforechapters и \nonbreakingbeforechapters
% управляют разрывом страницы перед главами.
% По-умолчанию страница разрывается.

% \nobreakingbeforechapters
% \breakingbeforechapters


\mainmatter % это включает нумерацию глав и секций в документе ниже

\tableofcontents

\chapter*{Введение}
\addcontentsline{toc}{chapter}{Введение}

Расстояние Левенштейна (редакционное расстояние, дистанция редактирования) — метрика, измеряющая по модулю разность между двумя последовательностями символов. Она определяется как минимальное количество односимвольных операций (а именно вставки, удаления, замены), необходимых для превращения одной последовательности символов в другую. В общем случае, операциям, используемым в этом преобразовании, можно назначить разные цены. Широко используется в теории информации и компьютерной лингвистике.

Впервые задачу поставил в 1965 году советский математик Владимир Левенштейн при изучении последовательностей \(0 − 1\) \cite{leven1965}, впоследствии более общую задачу для произвольного алфавита связали с его именем.

Расстояние Левентшейна используется:

\begin{itemize}
    \item в компьютерной лингвистике для проведения автозамены
    \item в биоинформатике для сравнения генов, хромосом, белков
    \item в утилите diff для определения изменений в текстовых файлов
\end{itemize}

Расстояние Дамерау — Левенштейна (названо в честь учёных Фредерика Дамерау и Владимира Левенштейна) — это мера разницы двух строк символов, определяемая как минимальное количество операций вставки, удаления, замены и транспозиции (перестановки двух соседних символов), необходимых для перевода одной строки в другую. Является модификацией расстояния Левенштейна: к операциям вставки, удаления и замены символов, определённых в расстоянии Левенштейна добавлена операция транспозиции (перестановки) символов.

Целью работы является исследование алгоритмов Левенштейна и Дамераy - Левенштейна.

Для достижения поставленной цели необходимо выполнить следующие задачи:

\begin{enumerate}
    \item изучение алгоритма Левенштейна и Демаран - Левенштейна
    \item реализация рекурсивных (без кеша) и итеративных (с кешем) алгоритмов Левенштейна и Дамераy - Левенштейна
    \item замер и сравнение процессорного времени затрачиваемых реализованными алгоритмами
    \item замер и сравнение используемой памяти реализованными алгоритмами
\end{enumerate}
\chapter{Аналитическая часть}

    В данном разделе описаны математические модели исследуемой области.

    \section{Описание задачи}

        Лексикографический порядок — отношение линейного порядка на множестве слов над некоторым упорядоченным алфавитом \( \sum \) . Своё название лексикографический порядок получил по аналогии с сортировкой по алфавиту в словаре. 
        
        \subsection*{Определение}
        
            Слово \( \alpha \) предшествует слову \( \beta \) (\( \alpha \) < \( \beta \)), если
            
            \begin{itemize}
                \item  либо первые \( m \) символов этих слов совпадают, а \( m+1 \)-й символ слова \( \alpha \) меньше (относительно заданного в \( \sum \) порядка) \( m + 1 \)-го символа слова \( \beta \) (например, АБАК < АБРАКАДАБРА, так как первые две буквы у этих слов совпадают, а третья буква у первого слова меньше, чем у второго);
                \item  либо слово \( \alpha \) является началом слова \( \beta \) (например, МАТЕМАТИК < МАТЕМАТИКА).
            \end{itemize}

        \subsection*{Примеры}

            Порядок слов в словаре. Предполагается, что буквы можно сравнивать, сравнивая их номера в алфавите. Например, следующие слова идут в лексикографическом порядке: А < АА < ААА < ААБ < ААВ < АБ < Б < … < ЯЯЯ.
    
            Естественный порядок на неотрицательных целых \(n\)-значных числах в любой позиционной системе счисления, записанных в разрядной сетке фиксированной длины (000, 001, 002, 003, 004, 005, …, 998, 999).
    
    \section*{Вывод}
    \addcontentsline{toc}{section}{Вывод}
    
        В результате были рассмотрены и описаны математические модели следующих алгоритмов:
        
        \begin{itemize}
            \item Последовательный алгоритм лексикографического сравнения строк
            \item Параллельный алгоритм лексикографического сравнения строк
        \end{itemize}
        
        В качестве входных данных алгоритмы принимают 2 строки. Выходными данными алгоритма является число. Если число равно нулю, то строки равны, если число больше нуля, то первая строка больше второй иначе вторая строка больше первой.
        
        Ограничением для реализуемой программы является обработка строк только в кодах ascii.c
\chapter{Конструкторская часть}

    В данном разделе представлены схемы рассматриваемых алгоримтов и их оценка по памяти.
    
    \section{Схемы алгоритмов}
    
        \begin{figure}
            \centering
            \includegraphics[width=15cm,height=25cm,keepaspectratio]{images/leven_rec.pdf}
            \caption{Рекурсивный алгоритм Левенштейна}
            \label{fig:leven_rec}
        \end{figure}
        
        \begin{figure}
            \centering
            \includegraphics[width=15cm,height=25cm,keepaspectratio]{images/leven_iter.pdf}
            \caption{Рекурсивный алгоритм Левенштейна}
            \label{fig:leven_iter}
        \end{figure}
        
        \begin{figure}
            \centering
            \includegraphics[width=15cm,height=25cm,keepaspectratio]{images/dleven_rec.pdf}
            \caption{Итеративный алгоритм Дамерау - Левенштейна}
            \label{fig:dleven_rec}
        \end{figure}
        
        \begin{figure}
            \centering
            \includegraphics[width=15cm,height=25cm,keepaspectratio]{images/dleven_iter.pdf}
            \caption{Рекурсивный алгоритм Дамерау - Левенштейна}
            \label{fig:dleven_iter}
        \end{figure}
    
    \clearpage
    
    \section{Оценка памяти}
    
        Алгоритмы Левенштейна и Дамерау — Левенштейна не отличаются друг от друга с точки зрения использования памяти, следовательно, достаточно рассмотреть лишь разницу рекурсивной и матричной реализаций этих алгоритмов.
        
        Максимальная глубина стека вызовов при рекурсивной реализации равна сумме длин входящих строк, соответственно, максимальный расход памяти (\ref{for:99})
        \begin{equation}
        (\mathcal{C}(S_1) + \mathcal{C}(S_2)) \cdot (2 \cdot \mathcal{C}\mathrm{(string)} + 3 \cdot \mathcal{C}\mathrm{(int)}),
        \label{for:99}
        \end{equation}
        где $\mathcal{C}$ — оператор вычисления размера, $S_1$, $S_2$ — строки, $\mathrm{int}$ — целочисленный тип, $\mathrm{string}$ — строковый тип.
        
        Использование памяти при итеративной реализации теоретически равно
        \begin{equation}
        (\mathcal{C}(S_1) + 1) \cdot (\mathcal{C}(S_2) + 1) \cdot \mathcal{C}\mathrm{(int)} + 10\cdot \mathcal{C}\mathrm{(int)} + 2 \cdot \mathcal{C}\mathrm{(string)}.
        \end{equation}

    \section{Структуры данных}
    
        В качестве входных данных алгоритмы принимают 2 строки, по которым рассчитывается искомое расстояние. Выходными данными алгоритма является число - найденное расстояние.
        
        В качестве строк для реализации алгоритма выберем нуль-терминировнаныее строки. Никаких других структур данных не потребуется.
        
    \section*{Вывод}
    \addcontentsline{toc}{section}{Вывод}
    
        На основе теоретических данных, полученных из аналитического раздела были построены схемы требуемых алгоритмов.
\chapter{Технологическая часть}

    В данном разделе приведены требования к программному обеспечению, средства реализации и листинги кода.
    
    \section{Выбор средств реализации}
        
        Для реализации библиотеки с реализациями алгоритмов мною был выбрал язык C++\cite{cppstd} стандарта 2014 года. Он был выбран в силу предоставления библиотеки работы с физическими потоками.
        
        Для юнит тестирования был выбран язык C++14, для использовая фрейворка GoogleTests\cite{gtests}.
    
    \section{Требования к программному обеспечению}
    
        К программе предъявляется ряд требований:
        
        \begin{itemize}
            \item На вход подаётся две регистрозависимые строки.
            \item На выходе — результат выполнения каждого из вышеуказанных алгоритмов.
        \end{itemize}
    
    \clearpage
    
    \section{Листинги кода}
    
        \lstinputlisting[
        language={C++},
        caption=Основной файл программы main.cpp,
        basicstyle=\scriptsize
        ]{./listings/main.cpp}
        
        \lstinputlisting[
        language={C++},
        caption=Последовательная реализация алгоритма,
        basicstyle=\scriptsize
        ]{./listings/nonparallel.cpp}
        
        \newpage
        
        \lstinputlisting[
        language={C++},
        caption=Параллельная реализация алгоритма,
        basicstyle=\scriptsize
        ]{./listings/parallel.cpp}
        
    \section*{Вывод}
    \addcontentsline{toc}{section}{Вывод}
    
        Были реализованы алгоритмы последовательного лексикографического сравнения строк и параллельного лексикографического сравнения строк.
\chapter{Экспериментальная часть}

    В данном разделе будет проведено функциональное тестирование разработанного программного обеспечения. Так же будет произведено измерение временных характеристик и характеристик по памяти каждого из реализованных алгоритмов.

    \section{Технические характеристики}
    
        Технические характеристики устройства, на котором выполнялось исследование:
        
        \begin{itemize}
            \item Процессор: Intel(R) Core(TM) i5-8250U CPU @ 1.60GHz \cite{intel}
            \item Оперативная память: 8 Gb
            \item Операционная система: Linux\cite{kernel} elementary OS 6 Odin \cite{elemos}
        \end{itemize}
    
    \section{Тестирование}
    
        Юнит тестирование проводилось при помощи фрэймворка GoogleTests\cite{gtests}. Были выполнены следующие тесты
        
        \begin{table}[]
            \centering
            \caption{Функциональные тесты}
            \begin{tabular}{|c|c|c|c|}
\hline \textbf{Строка 1} & \textbf{Строка 2} & \textbf{Последовательный} & \textbf{Параллельный}  \\
\hline &  & 0 & 0 \\
\hline Denis & Denis & 0 & 0 \\
\hline abc & bc & 1 & 1 \\
\hline bc & abc & -1 & -1 \\
\hline
            \end{tabular}
            \label{tab:my_label}
        \end{table}
        
        
        При проведении функционального тестирования, полученные результаты работы программы совпали с ожидаемыми. Таким образом, функциональное тестирование пройдено успешно.
    
    \section{Временные характеристики}
    
        Для сравнения временных характеристик проведем сравнение времени работы параллельного алгоритма для разного числа потоков, а так же сравнения последовательного и параллельного алгоритма.
        
        Данные эксперименты проведем для строк весом в 1, 2, 4, 8, 16, 32, 64 Мегабайта.
        
        Результаты замеров времени при разном количестве потоков приведены в таблице \ref{tab:time_by_th_num}.
        
        \begin{table}[]
            \centering
            \caption{Замер времени при разной длине строк и разном разном количестве потоков (время в миллисекундах)}
            \begin{tabular}{|c|c|c|c|c|c|c|c|}
\hline \multirow{2}{*}{\textbf{Вес строк, Mb}} &\multicolumn{7}{c|}{\textbf{Количество потоков, ед}} \\
\cline{2-8} &\textbf{1}&\textbf{2}&\textbf{4}&\textbf{8}&\textbf{16}&\textbf{32}&\textbf{64} \\
\hline 1&21&9&4&3&4&4&4 \\
\hline 2&33&18&14&8&7&7&8 \\
\hline 4&66&33&16&15&14&14&15 \\
\hline 8&132&66&34&29&33&30&31 \\
\hline 16&265&136&77&61&65&59&59 \\
\hline 32&536&267&151&130&127&120&118 \\
\hline 64&1055&532&266&233&241&245&236 \\
\hline 128&2117&1068&572&469&489&515&657 \\
\hline 256&4255&2134&1070&945&1159&1118&1325 \\
\hline 512&8475&4248&2751&2726&2707&2716&2695 \\
\hline
            \end{tabular}
            \label{tab:time_by_th_num}
        \end{table}
        
        Результаты замеров времени при выборе последовательного и параллельного алгоритма приведены в таблице \ref{tab:time_by_algorithm}.
        
        \begin{table}[]
            \centering
            \caption{Замер времени при разной длине строк и разных алгоритмах (время в миллисекундах)}
            \begin{tabular}{|c|c|c|}
\hline \textbf{Вес строк, Mb} & \textbf{Параллельный} & \textbf{Непараллельный} \\
\hline 1&5&22 \\
\hline 2&9&43 \\
\hline 4&21&85 \\
\hline 8&35&153 \\
\hline 16&64&285 \\
\hline 32&127&567 \\
\hline 64&283&1126 \\
\hline 128&570&2223 \\
\hline 256&1111&4419 \\
\hline 512&2342&8766 \\
\hline
            \end{tabular}
            \label{tab:time_by_algorithm}
        \end{table}
        
        
        \begin{figure}[!h]
          \centering
          \begin{tikzpicture}
            \begin{axis}[
              axis lines=left,
              xlabel={Вес строки, Mb},
              ylabel={Время, мс},
              legend pos=north west,
              ymajorgrids=true
            ]
              \addplot table[x=strsize_mb,y=parallel,col sep=comma] {datasheets/time_by_algorithm.csv};
              \addplot table[x=strsize_mb,y=nonparallel,col sep=comma] {datasheets/time_by_algorithm.csv};
              \legend{Параллельный, Последовательный}
            \end{axis}
          \end{tikzpicture}
          \captionsetup{justification=centering}
          \caption{Сравнение последовательного и параллельного алгоритма сравнения строк}
          \label{plt:time_by_algorithm}
        \end{figure}
        
        \begin{figure}[!h]
          \centering
          \begin{tikzpicture}
            \begin{axis}[
              axis lines=left,
              xlabel={Количество потоков, Mb},
              ylabel={Время, мс},
              legend pos=north west,
              ymajorgrids=true
            ]
              \addplot table[x=th_num,y=time,col sep=comma] {datasheets/time_th_num.csv};
            \end{axis}
          \end{tikzpicture}
          \captionsetup{justification=centering}
          \caption{Сравнение параллельного алгоритма при разном количестве потоков}
          \label{plt:time_th_num}
        \end{figure}
            
        На рисунке \ref{plt:time_by_algorithm} приведен график сравнения последовательного и параллельного алгоритма лексикографического сравнения строк при строка разной длины. Параллельный алгоритм использовател 8 потоков. Параллельный алгоритм работает в среднем в 4-5 раз быстрее последовательного.
        
        На рисунке \ref{plt:time_th_num} приведен график зависимости времени работы алгоритма от количества потоков для строки длины 256 Мегабайт. Наибольший выигрыш по времени происходит при количестве потоков равном 8, оно в 4.3 раза быстрее решения в 1 поток. При дальнейшем увеличении количества потоков время не уменьшается. На машине, на которой выполнялся данных эксперимент физически одновременно может выполнять 8 потоков. Следовательно, наибольший выигрыш по времени наблюдается при запуске максимального поддерживаемого процессором количества потоков, которые могут выполняться одновременно. 
    
    \section*{Вывод}
    \addcontentsline{toc}{section}{Вывод}
    
        В данном разделе было произведено сравнение количества затраченного времени и пaмяти вышеизложенных алгоритмов. 
        
        Паралельный алгоритм лексикографиеческого сравнения строк для количестве потоков равном количеству ядер процессора в среднем работает быстрее последовательного в 4-5 раз. Наибольшим выигрыш наблюдается при числе потоков равном максимальному числу потоков, которые могут физически выполнять одновременно. Дальнейнее увеличение числа потоков к росту производительности не ведет.
        
\chapter*{Заключение}
\addcontentsline{toc}{chapter}{Заключение}

    В результате выполнения данной лабораторной рабоыт были изучены алгоритмы лексикографического сравнения строк, построены схемы алгоритмов, реализованы данных алгоритмы и проведено их сравнение.
    
    Паралельный алгоритм лексикографиеческого сравнения строк для количестве потоков равном количеству ядер процессора в среднем работает быстрее последовательного в 4-5 раз. Наибольшим выигрыш наблюдается при числе потоков равном максимальному числу потоков, которые могут физически выполнять одновременно. Дальнейнее увеличение числа потоков к росту производительности не ведет.

\bibliographystyle{gost780u}
\bibliography{main.bib}

\end{document}

%%% Local Variables:
%%% mode: latex
%%% TeX-master: t
%%% End:
